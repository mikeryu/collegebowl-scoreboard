 \documentclass[11pt]{beamer}
\mode<presentation>
%\includeonlyframes{yep}


%% Packages
\usepackage{amsmath,amssymb,amsthm,amsfonts,graphicx,url,colordvi,bbm}
\usepackage{graphics,graphics,latexsym,multicol,epsfig}
\usepackage{enumerate,url}
\usepackage{wasysym}
 \usepackage{vwcol} 
\usepackage{pifont}
\usepackage{cancel}
\usepackage{empheq}

%% Color Definitions
% These are set with RGB codes, which you can for sure find online, but Macs have a cool app called Digital 
% Color Meter which is super helpful. I tend to give generic names so that I can just change the numbers
% when I want to change colors.
\definecolor{dark}{RGB}{30, 0, 70}
\definecolor{medium}{RGB}{129, 0, 49}
\definecolor{light}{RGB}{0, 0, 200}
\definecolor{accent}{RGB}{178,153,108}
\definecolor{ivory}{RGB}{255,255,255}


% \definecolor{ivory}{RGB}{250,240,225}


% \definecolor{accent}{RGB}{208,120,149}
% 178, 153, 108

%% Theme
% This is the style and layout. I use an outer theme that gets rid of slide numbers, which is just a style file 
% that you need to include wherever you've compiled your slides. I've attached it to the email.
% The navigation symbols are those things in the bottom right corner, but I don't like them so I set them to blank.
% \usetheme{Madrid}
% \usetheme{AnnArbor}
% \usetheme{Boadilla}
% \usetheme{Copenhagen}
% \usetheme{Darmstadt}
% \usetheme{Berkeley}
% \usetheme{Dresden}
% \usetheme{Frankfurt}
\usetheme{JuanLesPins}


\setbeamertemplate{navigation symbols}{}
\useoutertheme{noslidenum}
\setbeamersize{text margin left=15pt,text margin right=15pt}


%% Color Sets
% this is where I custom set the colors of all my objects. Beamer is weird in that most objects have a foreground (fg) color 
% and a background (bg) color. You can also do color mixing with exclamation points, so like the background canvas is 25% 
% white and 75% ivory.

\usecolortheme[named=medium]{structure}
\setbeamertemplate{enumerate items}[square]
\setbeamercolor{item projected}{bg=accent,fg=white}
\setbeamertemplate{itemize items}{\textcolor{accent}{$\RHD$}}
\setbeamertemplate{qed symbol}{\textcolor{accent}{$\darksquare$}}
\setbeamercolor{background canvas}{bg=ivory!10!white}
\setbeamercolor{author}{fg=dark}
\setbeamercolor{date}{fg=dark}
\setbeamercolor{institute}{fg=accent}
\setbeamercolor{normal text}{fg=dark}
\setbeamercolor{alerted text}{fg=accent}
\setbeamercolor{title}{bg=medium}
\setbeamercolor{title}{fg=white}
\setbeamercolor{block title}{bg=accent}
\setbeamercolor{block title}{fg=white}
\setbeamercolor{block body}{bg=white}
\setbeamercolor{section in head/foot}{fg=accent}
\setbeamercolor{institute in head/foot}{fg=accent}
\setbeamercolor{author in head/foot}{fg=accent}
\setbeamercolor{date in head/foot}{fg=accent}


%% Tikz
\usepackage{pgf,tikz}
\usetikzlibrary{arrows,snakes}
\usetikzlibrary{calc}

%% Math Commands
\newtheorem{conjecture}[theorem]{Conjecture}
\newtheorem{proposition}[theorem]{Proposition}
\newtheorem{remark}[theorem]{Remark}
\newtheorem{claim}{Claim}
\renewcommand{\L}{\mathcal{L}}
\newcommand*\widefbox[1]{\fbox{\hspace{1em}#1\hspace{1em}}}
\newcommand{\la}{\langle}
\newcommand{\ra}{\rangle}
\newcommand{\N}{\mathbb{N}}
\newcommand{\norm}[1]{\left\lVert#1\right\rVert}
\newcommand{\abs}[1]{|#1|}
\newcommand{\Abs}[1]{\big{|}#1\big{|}}
\newcommand{\ABS}[1]{\left|#1\right|}
\newcommand{\mch}{\mathcal{H}}
\def\({\left(}
\def\){\right)}

%% Font
% Fonts are complicated in Beamer, I always just end up googling it and copying/pasting
% the code they have listed. You can certainly delete this chunk if you want to use the default.
% \usepackage[english]{babel}
% %\usefonttheme{serif}
% \usepackage[light,math]{kurier}
% \usepackage[T1]{fontenc}
% \fontsize{11}{6}

\title[]{{11-12 College Bowl Preliminary Rounds - C}}
\author[]{{36th Annual Mathematics Field Day}}
\institute{Westmont College}
\date{2025}

\begin{document}

%%%%%%%%%%%%%%%%%%%%%%%%%%%%%%%%%%%%%%%%%%%%%%%%%%%%%%

\begin{frame}
  \titlepage
\end{frame}










%%%%%%%%%%%%%%%%%%%%%%%%%%%%%%%%%%%%%%%%%%%%%%%%%%%%%%
\section{Round 1}
\begin{frame}{}


{\LARGE
\begin{center}
Round 1
\end{center}
}
 
\end{frame}

%%%%%%%%%%%%%%%%%%%%%%%%%%%%%%%%%%%%%%%%%%%%%%%%%%%%%% R1 Question 1

\begin{frame}{}



{\LARGE
\begin{center}
Toss-Up Question
\end{center}}


 
\end{frame}

%%%%%%%%%%%%%%%%%%%%%%%%%%%%%%%%%%%%%%%%%%%%%%%%%%%%%%



\begin{frame}{}

{\LARGE
\begin{center}
    Write the expression $4\ln 3+2 \ln 5$ as a single natural logarithm without exponents.
\end{center}}

\end{frame}

%%%%%%%%%%%%%%%%%%%%%%%%%%%%%%%%%%%%%%%%%%%%%%%%%%%%%%



\begin{frame}{}


{\Large
\begin{center}
Answer 3

\pause

\vspace{1.5cm}

Answer 2

\pause 

\vspace{1.5cm}

Answer 1
\end{center}}

  
\end{frame}

%%%%%%%%%%%%%%%%%%%%%%%%%%%%%%%%%%%%%%%%%%%%%%%%%%%%%%


\begin{frame}{}


{\LARGE
\begin{center}
    $\ln 2025$
\end{center}}

\end{frame}



%%%%%%%%%%%%%%%%%%%%%%%%%%%%%%%%%%%%%%%%%%%%%%%%%%%%%%


\begin{frame}{}

{\LARGE
\begin{center}
Follow-Up Question
\end{center}}

\end{frame}


%%%%%%%%%%%%%%%%%%%%%%%%%%%%%%%%%%%%%%%%%%%%%%%%%%%%%%


\begin{frame}{}

{\LARGE
\begin{center}
    Simplify completely:
    $$
    \frac{1}{\log_{32} 10}+\frac{5}{\log_5 10}.
    $$
\end{center}}

\end{frame}


%%%%%%%%%%%%%%%%%%%%%%%%%%%%%%%%%%%%%%%%%%%%%%%%%%%%%%


\begin{frame}{}


{\Large
\begin{center}
Answer 3

\pause

\vspace{1.5cm}

Answer 2

\pause 

\vspace{1.5cm}

Answer 1
\end{center}}

  
\end{frame}


%%%%%%%%%%%%%%%%%%%%%%%%%%%%%%%%%%%%%%%%%%%%%%%%%%%%%%


\begin{frame}{}


{\LARGE
\begin{center}
    $5$
\end{center}}

\end{frame}


%%%%%%%%%%%%%%%%%%%%%%%%%%%%%%%%%%%%%%%



%%%%%%%%%%%%%%%%%%%%%%%%%%%%%%%%%%%%%%%%%%%%%%%%%%%%%% R1 Question 2

\begin{frame}{}




{\LARGE
\begin{center}
Toss-Up Question
\end{center}}


 
\end{frame}

%%%%%%%%%%%%%%%%%%%%%%%%%%%%%%%%%%%%%%%%%%%%%%%%%%%%%%



\begin{frame}{}

{\LARGE
\begin{center}
   Over a period of five years, the population of moose in a certain state increased by a factor of five. What was the population's constant annual rate of increase over this period?
\end{center}}

\end{frame}

%%%%%%%%%%%%%%%%%%%%%%%%%%%%%%%%%%%%%%%%%%%%%%%%%%%%%%



\begin{frame}{}


{\Large
\begin{center}
Answer 3

\pause

\vspace{1.5cm}

Answer 2

\pause 

\vspace{1.5cm}

Answer 1
\end{center}}

  
\end{frame}

%%%%%%%%%%%%%%%%%%%%%%%%%%%%%%%%%%%%%%%%%%%%%%%%%%%%%%


\begin{frame}{}


{\LARGE
\begin{center}
     $\sqrt[5]{5}-1\%$
\end{center}}

\end{frame}



%%%%%%%%%%%%%%%%%%%%%%%%%%%%%%%%%%%%%%%%%%%%%%%%%%%%%%


\begin{frame}{}

{\LARGE
\begin{center}
Follow-Up Question
\end{center}}

\end{frame}


%%%%%%%%%%%%%%%%%%%%%%%%%%%%%%%%%%%%%%%%%%%%%%%%%%%%%%


\begin{frame}{}

{\LARGE
\begin{center}
    The value of a certain car depreciates at a rate of 3\% each year. How long will it take for the car to lose half of its original value?
\end{center}}

\end{frame}


%%%%%%%%%%%%%%%%%%%%%%%%%%%%%%%%%%%%%%%%%%%%%%%%%%%%%%


\begin{frame}{}


{\Large
\begin{center}
Answer 3

\pause

\vspace{1.5cm}

Answer 2

\pause 

\vspace{1.5cm}

Answer 1
\end{center}}

  
\end{frame}


%%%%%%%%%%%%%%%%%%%%%%%%%%%%%%%%%%%%%%%%%%%%%%%%%%%%%%


\begin{frame}{}


{\LARGE
\begin{center}
    $-\frac{\ln 2}{\ln.97}$ years OR $\frac{\ln .5}{\ln.97}$ years
\end{center}}

\end{frame}


%%%%%%%%%%%%%%%%%%%%%%%%%%%%%%%%%%%%%%%%%%%%%%%%%%%%%%R1 Question 3
\begin{frame}{}


{\LARGE
\begin{center}
Toss-Up Question
\end{center}}


 
\end{frame}

%%%%%%%%%%%%%%%%%%%%%%%%%%%%%%%%%%%%%%%%%%%%%%%%%%%%%%



\begin{frame}{}

{\LARGE
\begin{center}
Suppose $a$, $b$, and $c$ are positive integers such that $a<b<c$ and
    $$
    \log a + \log b + \log c = 1.
    $$
    Find $a$, $b$, and $c$.
\end{center}}

\end{frame}

%%%%%%%%%%%%%%%%%%%%%%%%%%%%%%%%%%%%%%%%%%%%%%%%%%%%%%



\begin{frame}{}


{\Large
\begin{center}
Answer 3

\pause

\vspace{1.5cm}

Answer 2

\pause 

\vspace{1.5cm}

Answer 1
\end{center}}

  
\end{frame}

%%%%%%%%%%%%%%%%%%%%%%%%%%%%%%%%%%%%%%%%%%%%%%%%%%%%%%


\begin{frame}{}


{\LARGE
\begin{center}
    $a=1, b=2, c=5$
\end{center}}

\end{frame}



%%%%%%%%%%%%%%%%%%%%%%%%%%%%%%%%%%%%%%%%%%%%%%%%%%%%%%


\begin{frame}{}

{\LARGE
\begin{center}
Follow-Up Question
\end{center}}

\end{frame}


%%%%%%%%%%%%%%%%%%%%%%%%%%%%%%%%%%%%%%%%%%%%%%%%%%%%%%


\begin{frame}{}

{\LARGE
\begin{center}
    Find all solutions to $$1+\log_2(x^2-45)=3+\log_2 x.$$
\end{center}}

\end{frame}


%%%%%%%%%%%%%%%%%%%%%%%%%%%%%%%%%%%%%%%%%%%%%%%%%%%%%%


\begin{frame}{}


{\Large
\begin{center}
Answer 3

\pause

\vspace{1.5cm}

Answer 2

\pause 

\vspace{1.5cm}

Answer 1
\end{center}}

  
\end{frame}


%%%%%%%%%%%%%%%%%%%%%%%%%%%%%%%%%%%%%%%%%%%%%%%%%%%%%%


\begin{frame}{}

{\LARGE
\begin{center}
    $x=9$
\end{center}}

\end{frame}

%%%%%%%%%%%%%%%%%%%%%%%%%%%%%%%%%%%%%%%%%%%%%%%%%%%%%%


%%%%%%%%%%%%%%%%%%%%%%%%%%%%%%%%%%%%%%%%%%%%%%%%%%%%%% R1 Question 4
\begin{frame}{}




{\LARGE
\begin{center}
Toss-Up Question
\end{center}}


 
\end{frame}

%%%%%%%%%%%%%%%%%%%%%%%%%%%%%%%%%%%%%%%%%%%%%%%%%%%%%%



\begin{frame}{}

{\LARGE
\begin{center}
 Let $f(x)=e^{2x}$ and $g(x)=a\ln x$. Find $a$ such that $g(f(x))=x$ for all $x$.
\end{center}}

\end{frame}

%%%%%%%%%%%%%%%%%%%%%%%%%%%%%%%%%%%%%%%%%%%%%%%%%%%%%%



\begin{frame}{}


{\Large
\begin{center}
Answer 3

\pause

\vspace{1.5cm}

Answer 2

\pause 

\vspace{1.5cm}

Answer 1
\end{center}}

  
\end{frame}

%%%%%%%%%%%%%%%%%%%%%%%%%%%%%%%%%%%%%%%%%%%%%%%%%%%%%%


\begin{frame}{}


{\LARGE
\begin{center}
    $a=\frac{1}{2}$
\end{center}}

\end{frame}



%%%%%%%%%%%%%%%%%%%%%%%%%%%%%%%%%%%%%%%%%%%%%%%%%%%%%%


\begin{frame}{}

{\LARGE
\begin{center}
Follow-Up Question
\end{center}}

\end{frame}


%%%%%%%%%%%%%%%%%%%%%%%%%%%%%%%%%%%%%%%%%%%%%%%%%%%%%%


\begin{frame}{}

{\LARGE
\begin{center}
    Suppose $f(x)=\ln (x+4)-3$. Find the range of $f^{-1}(x)$.
\end{center}}

\end{frame}


%%%%%%%%%%%%%%%%%%%%%%%%%%%%%%%%%%%%%%%%%%%%%%%%%%%%%%


\begin{frame}{}


{\Large
\begin{center}
Answer 3

\pause

\vspace{1.5cm}

Answer 2

\pause 

\vspace{1.5cm}

Answer 1
\end{center}}

  
\end{frame}


%%%%%%%%%%%%%%%%%%%%%%%%%%%%%%%%%%%%%%%%%%%%%%%%%%%%%%


\begin{frame}{}


{\LARGE
\begin{center}
    $(-4,\infty)$
\end{center}}

\end{frame}

%%%%%%%%%%%%%%%%%%%%%%%%%%%%%%%%%%%%%%%%%%%%%%%%%%%%%%
\section{Round 2}
\begin{frame}{}


{\LARGE
\begin{center}
Round 2
\end{center}
}
 
\end{frame}

%%%%%%%%%%%%%%%%%%%%%%%%%%%%%%%%%%%%%%%%%%%%%%%%%%%%%% R2 Question 1
\begin{frame}{}


{\LARGE
\begin{center}
Toss-Up Question
\end{center}}


 
\end{frame}

%%%%%%%%%%%%%%%%%%%%%%%%%%%%%%%%%%%%%%%%%%%%%%%%%%%%%%



\begin{frame}{}

{\LARGE
\begin{center}
If $x$ is in the first quadrant and $\sin x = \frac{4}{5}$, find $\sin 2x$.
\end{center}}

\end{frame}

%%%%%%%%%%%%%%%%%%%%%%%%%%%%%%%%%%%%%%%%%%%%%%%%%%%%%%



\begin{frame}{}


{\Large
\begin{center}
Answer 3

\pause

\vspace{1.5cm}

Answer 2

\pause 

\vspace{1.5cm}

Answer 1
\end{center}}

  
\end{frame}

%%%%%%%%%%%%%%%%%%%%%%%%%%%%%%%%%%%%%%%%%%%%%%%%%%%%%%


\begin{frame}{}


{\LARGE
\begin{center}
    $\frac{24}{25}$
\end{center}}

\end{frame}



%%%%%%%%%%%%%%%%%%%%%%%%%%%%%%%%%%%%%%%%%%%%%%%%%%%%%%


\begin{frame}{}

{\LARGE
\begin{center}
Follow-Up Question
\end{center}}

\end{frame}


%%%%%%%%%%%%%%%%%%%%%%%%%%%%%%%%%%%%%%%%%%%%%%%%%%%%%%


\begin{frame}{}

{\LARGE
\begin{center}
    Find all solutions to the equation $\cos x = \cos 2x$ in the interval $[0,2\pi]$.
\end{center}}

\end{frame}


%%%%%%%%%%%%%%%%%%%%%%%%%%%%%%%%%%%%%%%%%%%%%%%%%%%%%%


\begin{frame}{}


{\Large
\begin{center}
Answer 3

\pause

\vspace{1.5cm}

Answer 2

\pause 

\vspace{1.5cm}

Answer 1
\end{center}}

  
\end{frame}


%%%%%%%%%%%%%%%%%%%%%%%%%%%%%%%%%%%%%%%%%%%%%%%%%%%%%%


\begin{frame}{}


{\LARGE
\begin{center}
    $0, \frac{2\pi}{3}$, $\frac{4\pi}{3}$, $2\pi$
\end{center}}

\end{frame}



%%%%%%%%%%%%%%%%%%%%%%%%%%%%%%%%%%%%%%%%%%%%%%%%%%%%%% R2 Question 2
\begin{frame}{}


{\LARGE
\begin{center}
Toss-Up Question
\end{center}}


 
\end{frame}

%%%%%%%%%%%%%%%%%%%%%%%%%%%%%%%%%%%%%%%%%%%%%%%%%%%%%%



\begin{frame}{}



{\LARGE
\begin{center}
Evaluate $1-\cos^2 x - \sin^2 x$.
\end{center}}


\end{frame}

%%%%%%%%%%%%%%%%%%%%%%%%%%%%%%%%%%%%%%%%%%%%%%%%%%%%%%



\begin{frame}{}


{\Large
\begin{center}
Answer 3

\pause

\vspace{1.5cm}

Answer 2

\pause 

\vspace{1.5cm}

Answer 1
\end{center}}

  
\end{frame}

%%%%%%%%%%%%%%%%%%%%%%%%%%%%%%%%%%%%%%%%%%%%%%%%%%%%%%


\begin{frame}{}

{\LARGE
\begin{center}
    $0$
\end{center}}

\end{frame}



%%%%%%%%%%%%%%%%%%%%%%%%%%%%%%%%%%%%%%%%%%%%%%%%%%%%%%


\begin{frame}{}

{\LARGE
\begin{center}
Follow-Up Question
\end{center}}

\end{frame}


%%%%%%%%%%%%%%%%%%%%%%%%%%%%%%%%%%%%%%%%%%%%%%%%%%%%%%


\begin{frame}{}

{\LARGE
\begin{center}
 Find all solutions to the following equation in the interval $[0,\pi]$.
    $$
    \frac{\sec^2\theta - 1}{\sec^2 \theta}=\frac{1}{4}
    $$  
\end{center}}

\end{frame}


%%%%%%%%%%%%%%%%%%%%%%%%%%%%%%%%%%%%%%%%%%%%%%%%%%%%%%


\begin{frame}{}


{\Large
\begin{center}
Answer 3

\pause

\vspace{1.5cm}

Answer 2

\pause 

\vspace{1.5cm}

Answer 1
\end{center}}

  
\end{frame}


%%%%%%%%%%%%%%%%%%%%%%%%%%%%%%%%%%%%%%%%%%%%%%%%%%%%%%


\begin{frame}{}


{\LARGE
\begin{center}
    $\frac{\pi}{6}$, $\frac{5\pi}{6}$
\end{center}}

\end{frame}

%%%%%%%%%%%%%%%%%%%%%%%%%%%%%%%%%%%%%%%%%%%%%%%%%%%%%%


%%%%%%%%%%%%%%%%%%%%%%%%%%%%%%%%%%%%%%%%%%%%%%%%%%%%%% R2 Question 3
\begin{frame}{}



{\LARGE
\begin{center}
Toss-Up Question
\end{center}}


 
\end{frame}

%%%%%%%%%%%%%%%%%%%%%%%%%%%%%%%%%%%%%%%%%%%%%%%%%%%%%%



\begin{frame}{}

{\LARGE
\begin{center}
Find an exact value of $\tan 105$\textdegree.
\end{center}}

\end{frame}

%%%%%%%%%%%%%%%%%%%%%%%%%%%%%%%%%%%%%%%%%%%%%%%%%%%%%%



\begin{frame}{}


{\Large
\begin{center}
Answer 3

\pause

\vspace{1.5cm}

Answer 2

\pause 

\vspace{1.5cm}

Answer 1
\end{center}}

  
\end{frame}

%%%%%%%%%%%%%%%%%%%%%%%%%%%%%%%%%%%%%%%%%%%%%%%%%%%%%%


\begin{frame}{}


{\LARGE
\begin{center}
    $-2-\sqrt{3}$
\end{center}}

\end{frame}



%%%%%%%%%%%%%%%%%%%%%%%%%%%%%%%%%%%%%%%%%%%%%%%%%%%%%%


\begin{frame}{}

{\LARGE
\begin{center}
Follow-Up Question
\end{center}}

\end{frame}


%%%%%%%%%%%%%%%%%%%%%%%%%%%%%%%%%%%%%%%%%%%%%%%%%%%%%%


\begin{frame}{}

{\LARGE
\begin{center}
Let $\theta$ be the acute angle at which the lines $y=3x-2$ and $y=2x+1$ intersect. Find $\tan \theta$.
\end{center}}

\end{frame}


%%%%%%%%%%%%%%%%%%%%%%%%%%%%%%%%%%%%%%%%%%%%%%%%%%%%%%


\begin{frame}{}


{\Large
\begin{center}
Answer 3

\pause

\vspace{1.5cm}

Answer 2

\pause 

\vspace{1.5cm}

Answer 1
\end{center}}

  
\end{frame}


%%%%%%%%%%%%%%%%%%%%%%%%%%%%%%%%%%%%%%%%%%%%%%%%%%%%%%


\begin{frame}{}


{\LARGE
\begin{center}
    $\frac{1}{7}$
\end{center}}

\end{frame}


%%%%%%%%%%%%%%%%%%%%%%%%%%%%%%%%%%%%%%%%%%%%%%%%%%%%%% R2 Question 4
\begin{frame}{}


{\LARGE
\begin{center}
Toss-Up Question
\end{center}}


 
\end{frame}

%%%%%%%%%%%%%%%%%%%%%%%%%%%%%%%%%%%%%%%%%%%%%%%%%%%%%%



\begin{frame}{}

{\LARGE
\begin{center}
If $\cos x = \frac{3}{4}$, what is $\sec (-x)$?
\end{center}}

\end{frame}

%%%%%%%%%%%%%%%%%%%%%%%%%%%%%%%%%%%%%%%%%%%%%%%%%%%%%%



\begin{frame}{}


{\Large
\begin{center}
Answer 3

\pause

\vspace{1.5cm}

Answer 2

\pause 

\vspace{1.5cm}

Answer 1
\end{center}}

  
\end{frame}

%%%%%%%%%%%%%%%%%%%%%%%%%%%%%%%%%%%%%%%%%%%%%%%%%%%%%%


\begin{frame}{}


{\LARGE
\begin{center}
    $\frac{4}{3}$
\end{center}}

\end{frame}



%%%%%%%%%%%%%%%%%%%%%%%%%%%%%%%%%%%%%%%%%%%%%%%%%%%%%%


\begin{frame}{}

{\LARGE
\begin{center}
Follow-Up Question
\end{center}}

\end{frame}


%%%%%%%%%%%%%%%%%%%%%%%%%%%%%%%%%%%%%%%%%%%%%%%%%%%%%%


\begin{frame}{}

{\LARGE
\begin{center}
Solve $(1+\sin x)(1+\sin (-x))=\frac{1}{4}$ for $x$ in the interval $[0,\pi]$.
\end{center}}

\end{frame}


%%%%%%%%%%%%%%%%%%%%%%%%%%%%%%%%%%%%%%%%%%%%%%%%%%%%%%


\begin{frame}{}


{\Large
\begin{center}
Answer 3

\pause

\vspace{1.5cm}

Answer 2

\pause 

\vspace{1.5cm}

Answer 1
\end{center}}

  
\end{frame}


%%%%%%%%%%%%%%%%%%%%%%%%%%%%%%%%%%%%%%%%%%%%%%%%%%%%%%


\begin{frame}{}


{\LARGE
\begin{center}
    $\frac{\pi}{3}$, $\frac{2\pi}{3}$
\end{center}}

\end{frame}

%%%%%%%%%%%%%%%%%%%%%%%%%%%%%%%%%%%%%%%%%%%%%%%%%%%%%%
\section{Round 3}
\begin{frame}{}


{\LARGE
\begin{center}
Round 3
\end{center}
}
 
\end{frame}

%%%%%%%%%%%%%%%%%%%%%%%%%%%%%%%%%%%%%%%%%%%%%%%%%%%%%% R3 Question 1
\begin{frame}{}


{\LARGE
\begin{center}
Toss-Up Question
\end{center}}


 
\end{frame}

%%%%%%%%%%%%%%%%%%%%%%%%%%%%%%%%%%%%%%%%%%%%%%%%%%%%%%



\begin{frame}{}

{\LARGE
\begin{center}
What is the area of the circle given by the equation $x^2-4x+y^2+2y-20=0$?
\end{center}}

\end{frame}

%%%%%%%%%%%%%%%%%%%%%%%%%%%%%%%%%%%%%%%%%%%%%%%%%%%%%%



\begin{frame}{}


{\Large
\begin{center}
Answer 3

\pause

\vspace{1.5cm}

Answer 2

\pause 

\vspace{1.5cm}

Answer 1
\end{center}}

  
\end{frame}

%%%%%%%%%%%%%%%%%%%%%%%%%%%%%%%%%%%%%%%%%%%%%%%%%%%%%%


\begin{frame}{}


{\LARGE
\begin{center}
    $25\pi$
\end{center}}

\end{frame}



%%%%%%%%%%%%%%%%%%%%%%%%%%%%%%%%%%%%%%%%%%%%%%%%%%%%%%


\begin{frame}{}

{\LARGE
\begin{center}
Follow-Up Question
\end{center}}

\end{frame}


%%%%%%%%%%%%%%%%%%%%%%%%%%%%%%%%%%%%%%%%%%%%%%%%%%%%%%


\begin{frame}{}

{\LARGE
\begin{center}
 Find the product of the length of the major and minor axes of the ellipse given by the equation $9x^2-18x+4y^2+8y=23$.
\end{center}}

\end{frame}


%%%%%%%%%%%%%%%%%%%%%%%%%%%%%%%%%%%%%%%%%%%%%%%%%%%%%%


\begin{frame}{}


{\Large
\begin{center}
Answer 3

\pause

\vspace{1.5cm}

Answer 2

\pause 

\vspace{1.5cm}

Answer 1
\end{center}}

  
\end{frame}


%%%%%%%%%%%%%%%%%%%%%%%%%%%%%%%%%%%%%%%%%%%%%%%%%%%%%%


\begin{frame}{}


{\LARGE
\begin{center}
    $24$
\end{center}}

\end{frame}



%%%%%%%%%%%%%%%%%%%%%%%%%%%%%%%%%%%%%%%%%%%%%%%%%%%%%% R3 Question 2
\begin{frame}{}



{\LARGE
\begin{center}
Toss-Up Question
\end{center}}


 
\end{frame}

%%%%%%%%%%%%%%%%%%%%%%%%%%%%%%%%%%%%%%%%%%%%%%%%%%%%%%



\begin{frame}{}



{\LARGE
\begin{center}
Recall that a parabola consists of all points equidistant from a point, called the focus, and a line, called the directrix. What is the equation of the directrix of the parabola $y=x^2$?
\end{center}}


\end{frame}

%%%%%%%%%%%%%%%%%%%%%%%%%%%%%%%%%%%%%%%%%%%%%%%%%%%%%%



\begin{frame}{}


{\Large
\begin{center}
Answer 3

\pause

\vspace{1.5cm}

Answer 2

\pause 

\vspace{1.5cm}

Answer 1
\end{center}}

  
\end{frame}

%%%%%%%%%%%%%%%%%%%%%%%%%%%%%%%%%%%%%%%%%%%%%%%%%%%%%%


\begin{frame}{}

{\LARGE
\begin{center}
    $y=-\frac{1}{4}$
\end{center}}

\end{frame}



%%%%%%%%%%%%%%%%%%%%%%%%%%%%%%%%%%%%%%%%%%%%%%%%%%%%%%


\begin{frame}{}

{\LARGE
\begin{center}
Follow-Up Question
\end{center}}

\end{frame}


%%%%%%%%%%%%%%%%%%%%%%%%%%%%%%%%%%%%%%%%%%%%%%%%%%%%%%


\begin{frame}{}

{\LARGE
\begin{center}
 A searchlight is made in the shape of a parabola with the light source placed at the focus. If the light source is $2$ feet from the vertex and the depth of the searchlight is $4$ feet, what should the width of the opening be? 
\end{center}}

\end{frame}


%%%%%%%%%%%%%%%%%%%%%%%%%%%%%%%%%%%%%%%%%%%%%%%%%%%%%%


\begin{frame}{}


{\Large
\begin{center}
Answer 3

\pause

\vspace{1.5cm}

Answer 2

\pause 

\vspace{1.5cm}

Answer 1
\end{center}}

  
\end{frame}


%%%%%%%%%%%%%%%%%%%%%%%%%%%%%%%%%%%%%%%%%%%%%%%%%%%%%%


\begin{frame}{}


{\LARGE
\begin{center}
    $8\sqrt{2}$
\end{center}}

\end{frame}

%%%%%%%%%%%%%%%%%%%%%%%%%%%%%%%%%%%%%%%%%%%%%%%%%%%%%%


%%%%%%%%%%%%%%%%%%%%%%%%%%%%%%%%%%%%%%%%%%%%%%%%%%%%%% R3 Question 3
\begin{frame}{}


{\LARGE
\begin{center}
Toss-Up Question
\end{center}}


\end{frame}

%%%%%%%%%%%%%%%%%%%%%%%%%%%%%%%%%%%%%%%%%%%%%%%%%%%%%%



\begin{frame}{}

{\LARGE
\begin{center}
Find the area of the ellipse given by the equation $2x^2+y^2=10$.
\end{center}}

\end{frame}

%%%%%%%%%%%%%%%%%%%%%%%%%%%%%%%%%%%%%%%%%%%%%%%%%%%%%%



\begin{frame}{}


{\Large
\begin{center}
Answer 3

\pause

\vspace{1.5cm}

Answer 2

\pause 

\vspace{1.5cm}

Answer 1
\end{center}}

  
\end{frame}

%%%%%%%%%%%%%%%%%%%%%%%%%%%%%%%%%%%%%%%%%%%%%%%%%%%%%%


\begin{frame}{}


{\LARGE
\begin{center}
    $5\sqrt{2}\pi$
\end{center}}

\end{frame}



%%%%%%%%%%%%%%%%%%%%%%%%%%%%%%%%%%%%%%%%%%%%%%%%%%%%%%


\begin{frame}{}

{\LARGE
\begin{center}
Follow-Up Question
\end{center}}

\end{frame}


%%%%%%%%%%%%%%%%%%%%%%%%%%%%%%%%%%%%%%%%%%%%%%%%%%%%%%


\begin{frame}{}

{\LARGE
\begin{center}
Find the equation of the ellipse with foci at $(1, 3)$ and $(9,3)$ and a minor axis of length $8$. 
\end{center}}

\end{frame}


%%%%%%%%%%%%%%%%%%%%%%%%%%%%%%%%%%%%%%%%%%%%%%%%%%%%%%


\begin{frame}{}


{\Large
\begin{center}
Answer 3

\pause

\vspace{1.5cm}

Answer 2

\pause 

\vspace{1.5cm}

Answer 1
\end{center}}

  
\end{frame}


%%%%%%%%%%%%%%%%%%%%%%%%%%%%%%%%%%%%%%%%%%%%%%%%%%%%%%


\begin{frame}{}


{\LARGE
\begin{center}
    $\frac{(x-5)^2}{32}+\frac{(y-3)^2}{16}=1$
\end{center}}

\end{frame}


%%%%%%%%%%%%%%%%%%%%%%%%%%%%%%%%%%%%%%%%%%%%%%%%%%%%%% R3 Question 4
\begin{frame}{}



{\LARGE
\begin{center}
Toss-Up Question
\end{center}}


 
\end{frame}

%%%%%%%%%%%%%%%%%%%%%%%%%%%%%%%%%%%%%%%%%%%%%%%%%%%%%%



\begin{frame}{}

{\LARGE
\begin{center}
A boat follows a hyperbolic path with Island A and Island B as foci. At 10:00am the boat is $10$ miles from Island A and $15$ miles from Island B. At 2:00pm the boat is $8$ miles away from Island A. What are all possible distances between the boat and Island B at 2:00pm?
\end{center}}

\end{frame}

%%%%%%%%%%%%%%%%%%%%%%%%%%%%%%%%%%%%%%%%%%%%%%%%%%%%%%



\begin{frame}{}


{\Large
\begin{center}
Answer 3

\pause

\vspace{1.5cm}

Answer 2

\pause 

\vspace{1.5cm}

Answer 1
\end{center}}

  
\end{frame}

%%%%%%%%%%%%%%%%%%%%%%%%%%%%%%%%%%%%%%%%%%%%%%%%%%%%%%


\begin{frame}{}


{\LARGE
\begin{center}
    $13$ miles or $3$ miles
\end{center}}

\end{frame}



%%%%%%%%%%%%%%%%%%%%%%%%%%%%%%%%%%%%%%%%%%%%%%%%%%%%%%


\begin{frame}{}

{\LARGE
\begin{center}
Follow-Up Question
\end{center}}

\end{frame}


%%%%%%%%%%%%%%%%%%%%%%%%%%%%%%%%%%%%%%%%%%%%%%%%%%%%%%


\begin{frame}{}

{\LARGE
\begin{center}
Find the standard form of a hyperbola symmetric about the $y$-axis with one vertex at $(45,0)$ and one focus at $(\sqrt{2061}, 0)$.
\end{center}}

\end{frame}


%%%%%%%%%%%%%%%%%%%%%%%%%%%%%%%%%%%%%%%%%%%%%%%%%%%%%%


\begin{frame}{}


{\Large
\begin{center}
Answer 3

\pause

\vspace{1.5cm}

Answer 2

\pause 

\vspace{1.5cm}

Answer 1
\end{center}}

  
\end{frame}


%%%%%%%%%%%%%%%%%%%%%%%%%%%%%%%%%%%%%%%%%%%%%%%%%%%%%%


\begin{frame}{}


{\LARGE
\begin{center}
   $\frac{x^2}{2025}-\frac{y^2}{36}=1$
\end{center}}

\end{frame}


%%%%%%%%%%%%%%%%%%%%%%%%%%%%%%%%%%%%%%%%%%%%%%%%%%%%%%
\section{Round 4}
\begin{frame}{}


{\LARGE
\begin{center}
Round 4
\end{center}
}
 
\end{frame}

%%%%%%%%%%%%%%%%%%%%%%%%%%%%%%%%%%%%%%%%%%%%%%%%%%%%%% R4 Question 1
\begin{frame}{}


{\LARGE
\begin{center}
Toss-Up Question
\end{center}}


\end{frame}

%%%%%%%%%%%%%%%%%%%%%%%%%%%%%%%%%%%%%%%%%%%%%%%%%%%%%%



\begin{frame}{}

{\LARGE
\begin{center}
What is the probability that a factor of $2025$ is also a multiple of $5$?
\end{center}}

\end{frame}

%%%%%%%%%%%%%%%%%%%%%%%%%%%%%%%%%%%%%%%%%%%%%%%%%%%%%%



\begin{frame}{}


{\Large
\begin{center}
Answer 3

\pause

\vspace{1.5cm}

Answer 2

\pause 

\vspace{1.5cm}

Answer 1
\end{center}}

  
\end{frame}

%%%%%%%%%%%%%%%%%%%%%%%%%%%%%%%%%%%%%%%%%%%%%%%%%%%%%%


\begin{frame}{}


{\LARGE
\begin{center}
    $\frac{2}{3}$ OR $66\frac{2}{3}\%$ OR $.\bar{6}$
\end{center}}

\end{frame}



%%%%%%%%%%%%%%%%%%%%%%%%%%%%%%%%%%%%%%%%%%%%%%%%%%%%%%


\begin{frame}{}

{\LARGE
\begin{center}
Follow-Up Question
\end{center}}

\end{frame}


%%%%%%%%%%%%%%%%%%%%%%%%%%%%%%%%%%%%%%%%%%%%%%%%%%%%%%


\begin{frame}{}

{\LARGE
\begin{center}
    Suppose that $x$ is a factor of 2025. Given that $x$ is a multiple of $5$, what is the probability $x$ is also a multiple of $25$?
\end{center}}

\end{frame}


%%%%%%%%%%%%%%%%%%%%%%%%%%%%%%%%%%%%%%%%%%%%%%%%%%%%%%


\begin{frame}{}


{\Large
\begin{center}
Answer 3

\pause

\vspace{1.5cm}

Answer 2

\pause 

\vspace{1.5cm}

Answer 1
\end{center}}

  
\end{frame}


%%%%%%%%%%%%%%%%%%%%%%%%%%%%%%%%%%%%%%%%%%%%%%%%%%%%%%


\begin{frame}{}


{\LARGE
\begin{center}
    $\frac{1}{2}$
\end{center}}

\end{frame}



%%%%%%%%%%%%%%%%%%%%%%%%%%%%%%%%%%%%%%%%%%%%%%%%%%%%%% R4 Question 2
\begin{frame}{}


{\LARGE
\begin{center}
Toss-Up Question
\end{center}}


 
\end{frame}

%%%%%%%%%%%%%%%%%%%%%%%%%%%%%%%%%%%%%%%%%%%%%%%%%%%%%%



\begin{frame}{}



{\LARGE
\begin{center}
A committee of three is chosen at random from a class with $10$ students. How many distinct committees can be chosen?
\end{center}}


\end{frame}

%%%%%%%%%%%%%%%%%%%%%%%%%%%%%%%%%%%%%%%%%%%%%%%%%%%%%%



\begin{frame}{}


{\Large
\begin{center}
Answer 3

\pause

\vspace{1.5cm}

Answer 2

\pause 

\vspace{1.5cm}

Answer 1
\end{center}}

  
\end{frame}

%%%%%%%%%%%%%%%%%%%%%%%%%%%%%%%%%%%%%%%%%%%%%%%%%%%%%%


\begin{frame}{}

{\LARGE
\begin{center}
    $120$
\end{center}}

\end{frame}



%%%%%%%%%%%%%%%%%%%%%%%%%%%%%%%%%%%%%%%%%%%%%%%%%%%%%%


\begin{frame}{}

{\LARGE
\begin{center}
Follow-Up Question
\end{center}}

\end{frame}


%%%%%%%%%%%%%%%%%%%%%%%%%%%%%%%%%%%%%%%%%%%%%%%%%%%%%%


\begin{frame}{}

{\LARGE
\begin{center}
A committee of three is chosen at random from a class of $7$ boys and $3$ girls. What is the probability that the committee will have at least one girl?
\end{center}}

\end{frame}


%%%%%%%%%%%%%%%%%%%%%%%%%%%%%%%%%%%%%%%%%%%%%%%%%%%%%%


\begin{frame}{}


{\Large
\begin{center}
Answer 3

\pause

\vspace{1.5cm}

Answer 2

\pause 

\vspace{1.5cm}

Answer 1
\end{center}}

  
\end{frame}


%%%%%%%%%%%%%%%%%%%%%%%%%%%%%%%%%%%%%%%%%%%%%%%%%%%%%%


\begin{frame}{}


{\LARGE
\begin{center}
    $\frac{17}{24}$
\end{center}}

\end{frame}

%%%%%%%%%%%%%%%%%%%%%%%%%%%%%%%%%%%%%%%%%%%%%%%%%%%%%%


%%%%%%%%%%%%%%%%%%%%%%%%%%%%%%%%%%%%%%%%%%%%%%%%%%%%%% R4 Question 3
\begin{frame}{}


{\LARGE
\begin{center}
Toss-Up Question
\end{center}}


 
\end{frame}

%%%%%%%%%%%%%%%%%%%%%%%%%%%%%%%%%%%%%%%%%%%%%%%%%%%%%%



\begin{frame}{}

{\LARGE
\begin{center}
Suppose that $x$ and $y$ are positive integers such that $x+y<10$. What is the probability that $x=8$?
\end{center}}

\end{frame}

%%%%%%%%%%%%%%%%%%%%%%%%%%%%%%%%%%%%%%%%%%%%%%%%%%%%%%



\begin{frame}{}


{\Large
\begin{center}
Answer 3

\pause

\vspace{1.5cm}

Answer 2

\pause 

\vspace{1.5cm}

Answer 1
\end{center}}

  
\end{frame}

%%%%%%%%%%%%%%%%%%%%%%%%%%%%%%%%%%%%%%%%%%%%%%%%%%%%%%


\begin{frame}{}


{\LARGE
\begin{center}
    $\frac{1}{36}$
\end{center}}

\end{frame}



%%%%%%%%%%%%%%%%%%%%%%%%%%%%%%%%%%%%%%%%%%%%%%%%%%%%%%


\begin{frame}{}

{\LARGE
\begin{center}
Follow-Up Question
\end{center}}

\end{frame}


%%%%%%%%%%%%%%%%%%%%%%%%%%%%%%%%%%%%%%%%%%%%%%%%%%%%%%


\begin{frame}{}

{\LARGE
\begin{center}
Suppose that $x^2+y^2<1$. What is the probability that $x+y>1$?
\end{center}}

\end{frame}


%%%%%%%%%%%%%%%%%%%%%%%%%%%%%%%%%%%%%%%%%%%%%%%%%%%%%%


\begin{frame}{}


{\Large
\begin{center}
Answer 3

\pause

\vspace{1.5cm}

Answer 2

\pause 

\vspace{1.5cm}

Answer 1
\end{center}}

  
\end{frame}


%%%%%%%%%%%%%%%%%%%%%%%%%%%%%%%%%%%%%%%%%%%%%%%%%%%%%%


\begin{frame}{}


{\LARGE
\begin{center}
    $\frac{\pi-2}{4\pi}$ OR $\frac{1}{4}-\frac{1}{2\pi}$
\end{center}}

\end{frame}


%%%%%%%%%%%%%%%%%%%%%%%%%%%%%%%%%%%%%%%%%%%%%%%%%%%%%% R4 Question 4
\begin{frame}{}


{\LARGE
\begin{center}
Toss-Up Question
\end{center}}



 
\end{frame}

%%%%%%%%%%%%%%%%%%%%%%%%%%%%%%%%%%%%%%%%%%%%%%%%%%%%%%



\begin{frame}{}

{\LARGE
\begin{center}
Alaska license plates consist of three letters followed by three digits. How many license plates are possible if numbers and letters may be repeated? You may leave exponents in your answer.
\end{center}}

\end{frame}

%%%%%%%%%%%%%%%%%%%%%%%%%%%%%%%%%%%%%%%%%%%%%%%%%%%%%%



\begin{frame}{}


{\Large
\begin{center}
Answer 3

\pause

\vspace{1.5cm}

Answer 2

\pause 

\vspace{1.5cm}

Answer 1
\end{center}}

  
\end{frame}

%%%%%%%%%%%%%%%%%%%%%%%%%%%%%%%%%%%%%%%%%%%%%%%%%%%%%%


\begin{frame}{}


{\LARGE
\begin{center}
    $26^3\cdot 10^3$ OR $260^3$ OR $17,576,000$ OR $26^3\cdot 1000$
\end{center}}

\end{frame}



%%%%%%%%%%%%%%%%%%%%%%%%%%%%%%%%%%%%%%%%%%%%%%%%%%%%%%


\begin{frame}{}

{\LARGE
\begin{center}
Follow-Up Question
\end{center}}

\end{frame}


%%%%%%%%%%%%%%%%%%%%%%%%%%%%%%%%%%%%%%%%%%%%%%%%%%%%%%


\begin{frame}{}

{\LARGE
\begin{center}
What is the probability that an Alaska license plate chosen at random will have at least one duplicate number or letter? Simplify your answer fully.
\end{center}}

\end{frame}


%%%%%%%%%%%%%%%%%%%%%%%%%%%%%%%%%%%%%%%%%%%%%%%%%%%%%%


\begin{frame}{}


{\Large
\begin{center}
Answer 3

\pause

\vspace{1.5cm}

Answer 2

\pause 

\vspace{1.5cm}

Answer 1
\end{center}}

  
\end{frame}


%%%%%%%%%%%%%%%%%%%%%%%%%%%%%%%%%%%%%%%%%%%%%%%%%%%%%%


\begin{frame}{}


{\LARGE
\begin{center}
    $\frac{61}{169}$
\end{center}}

\end{frame}


%%%%%%%%%%%%%%%%%%%%%%%%%%%%%%%%%%%%%%%%%%%%%%%%%%%%%%
\section{Round 5 - Calc}
\begin{frame}{}



{\LARGE
\begin{center}
Round 5
\end{center}
}
 
\end{frame}

%%%%%%%%%%%%%%%%%%%%%%%%%%%%%%%%%%%%%%%%%%%%%%%%%%%%%% R5 Question 1
\begin{frame}{}


{\LARGE
\begin{center}
Toss-Up Question
\end{center}}


 
\end{frame}

%%%%%%%%%%%%%%%%%%%%%%%%%%%%%%%%%%%%%%%%%%%%%%%%%%%%%%



\begin{frame}{}

{\LARGE
\begin{center}
Find the third derivative of $e^{2x}$.
\end{center}}

\end{frame}

%%%%%%%%%%%%%%%%%%%%%%%%%%%%%%%%%%%%%%%%%%%%%%%%%%%%%%



\begin{frame}{}


{\Large
\begin{center}
Answer 3

\pause

\vspace{1.5cm}

Answer 2

\pause 

\vspace{1.5cm}

Answer 1
\end{center}}

  
\end{frame}

%%%%%%%%%%%%%%%%%%%%%%%%%%%%%%%%%%%%%%%%%%%%%%%%%%%%%%


\begin{frame}{}


{\LARGE
\begin{center}
    $8e^{2x}$
\end{center}}

\end{frame}



%%%%%%%%%%%%%%%%%%%%%%%%%%%%%%%%%%%%%%%%%%%%%%%%%%%%%%


\begin{frame}{}

{\LARGE
\begin{center}
Follow-Up Question
\end{center}}

\end{frame}


%%%%%%%%%%%%%%%%%%%%%%%%%%%%%%%%%%%%%%%%%%%%%%%%%%%%%%


\begin{frame}{}

{\LARGE
\begin{center}
    Let $f(x)=xe^{x^2}$. Find the slope of the line tangent to $f'(x)$ at $x=1$. (You may leave your answer in terms of $e$).
\end{center}}

\end{frame}


%%%%%%%%%%%%%%%%%%%%%%%%%%%%%%%%%%%%%%%%%%%%%%%%%%%%%%


\begin{frame}{}


{\Large
\begin{center}
Answer 3

\pause

\vspace{1.5cm}

Answer 2

\pause 

\vspace{1.5cm}

Answer 1
\end{center}}

  
\end{frame}


%%%%%%%%%%%%%%%%%%%%%%%%%%%%%%%%%%%%%%%%%%%%%%%%%%%%%%


\begin{frame}{}


{\LARGE
\begin{center}
    $10e$
\end{center}}

\end{frame}



%%%%%%%%%%%%%%%%%%%%%%%%%%%%%%%%%%%%%%%%%%%%%%%%%%%%%% R5 Question 2
\begin{frame}{}


{\LARGE
\begin{center}
Toss-Up Question
\end{center}}


 
\end{frame}

%%%%%%%%%%%%%%%%%%%%%%%%%%%%%%%%%%%%%%%%%%%%%%%%%%%%%%



\begin{frame}{}



{\LARGE
\begin{center}
Two cars start at the same point. The first car drives North at a speed of $10$ miles per hour and the second drives East at a speed of $20$ miles per hour. What is the distance between the two cars after three hours?
\end{center}}


\end{frame}

%%%%%%%%%%%%%%%%%%%%%%%%%%%%%%%%%%%%%%%%%%%%%%%%%%%%%%



\begin{frame}{}


{\Large
\begin{center}
Answer 3

\pause

\vspace{1.5cm}

Answer 2

\pause 

\vspace{1.5cm}

Answer 1
\end{center}}

  
\end{frame}

%%%%%%%%%%%%%%%%%%%%%%%%%%%%%%%%%%%%%%%%%%%%%%%%%%%%%%


\begin{frame}{}

{\LARGE
\begin{center}
    $30\sqrt{5}$ miles
\end{center}}

\end{frame}



%%%%%%%%%%%%%%%%%%%%%%%%%%%%%%%%%%%%%%%%%%%%%%%%%%%%%%


\begin{frame}{}

{\LARGE
\begin{center}
Follow-Up Question
\end{center}}

\end{frame}


%%%%%%%%%%%%%%%%%%%%%%%%%%%%%%%%%%%%%%%%%%%%%%%%%%%%%%


\begin{frame}{}

{\LARGE
\begin{center}
 Two cars start at the same point. The first car drives North at a speed of $10$ miles per hour and the second drives East at a speed of $20$ miles per hour. At what rate is the distance between the two cars changing? 
\end{center}}

\end{frame}


%%%%%%%%%%%%%%%%%%%%%%%%%%%%%%%%%%%%%%%%%%%%%%%%%%%%%%


\begin{frame}{}


{\Large
\begin{center}
Answer 3

\pause

\vspace{1.5cm}

Answer 2

\pause 

\vspace{1.5cm}

Answer 1
\end{center}}

  
\end{frame}


%%%%%%%%%%%%%%%%%%%%%%%%%%%%%%%%%%%%%%%%%%%%%%%%%%%%%%


\begin{frame}{}


{\LARGE
\begin{center}
    $10\sqrt{5}$ miles per hour
\end{center}}

\end{frame}

%%%%%%%%%%%%%%%%%%%%%%%%%%%%%%%%%%%%%%%%%%%%%%%%%%%%%%


%%%%%%%%%%%%%%%%%%%%%%%%%%%%%%%%%%%%%%%%%%%%%%%%%%%%%% R5 Question 3
\begin{frame}{}


{\LARGE
\begin{center}
Toss-Up Question
\end{center}}


 
\end{frame}

%%%%%%%%%%%%%%%%%%%%%%%%%%%%%%%%%%%%%%%%%%%%%%%%%%%%%%



\begin{frame}{}

{\LARGE
\begin{center}
Evaluate

$$\lim_{x\to 4}\frac{\sin(\pi x)}{x-4}.$$
\end{center}}

\end{frame}

%%%%%%%%%%%%%%%%%%%%%%%%%%%%%%%%%%%%%%%%%%%%%%%%%%%%%%



\begin{frame}{}


{\Large
\begin{center}
Answer 3

\pause

\vspace{1.5cm}

Answer 2

\pause 

\vspace{1.5cm}

Answer 1
\end{center}}

  
\end{frame}

%%%%%%%%%%%%%%%%%%%%%%%%%%%%%%%%%%%%%%%%%%%%%%%%%%%%%%


\begin{frame}{}


{\LARGE
\begin{center}
    $\pi$
\end{center}}

\end{frame}



%%%%%%%%%%%%%%%%%%%%%%%%%%%%%%%%%%%%%%%%%%%%%%%%%%%%%%


\begin{frame}{}

{\LARGE
\begin{center}
Follow-Up Question
\end{center}}

\end{frame}


%%%%%%%%%%%%%%%%%%%%%%%%%%%%%%%%%%%%%%%%%%%%%%%%%%%%%%


\begin{frame}{}

{\LARGE
\begin{center}
Evaluate $$\lim_{x\to\infty}(e^x+x)^\frac{1}{x}.$$
\end{center}}

\end{frame}


%%%%%%%%%%%%%%%%%%%%%%%%%%%%%%%%%%%%%%%%%%%%%%%%%%%%%%


\begin{frame}{}


{\Large
\begin{center}
Answer 3

\pause

\vspace{1.5cm}

Answer 2

\pause 

\vspace{1.5cm}

Answer 1
\end{center}}

  
\end{frame}


%%%%%%%%%%%%%%%%%%%%%%%%%%%%%%%%%%%%%%%%%%%%%%%%%%%%%%


\begin{frame}{}


{\LARGE
\begin{center}
    $e$
\end{center}}

\end{frame}


%%%%%%%%%%%%%%%%%%%%%%%%%%%%%%%%%%%%%%%%%%%%%%%%%%%%%% R5 Question 4
\begin{frame}{}


{\LARGE
\begin{center}
Toss-Up Question
\end{center}}




 
\end{frame}

%%%%%%%%%%%%%%%%%%%%%%%%%%%%%%%%%%%%%%%%%%%%%%%%%%%%%%



\begin{frame}{}

{\LARGE
\begin{center}
Find the $x$-coordinates of all local maxima of the function $g(x)=\frac{1}{3}x^3-\frac{3}{2}x^2+2x+5.$
\end{center}}

\end{frame}

%%%%%%%%%%%%%%%%%%%%%%%%%%%%%%%%%%%%%%%%%%%%%%%%%%%%%%



\begin{frame}{}


{\Large
\begin{center}
Answer 3

\pause

\vspace{1.5cm}

Answer 2

\pause 

\vspace{1.5cm}

Answer 1
\end{center}}

  
\end{frame}

%%%%%%%%%%%%%%%%%%%%%%%%%%%%%%%%%%%%%%%%%%%%%%%%%%%%%%


\begin{frame}{}


{\LARGE
\begin{center}
    $x=1$
\end{center}}

\end{frame}



%%%%%%%%%%%%%%%%%%%%%%%%%%%%%%%%%%%%%%%%%%%%%%%%%%%%%%


\begin{frame}{}

{\LARGE
\begin{center}
Follow-Up Question
\end{center}}

\end{frame}


%%%%%%%%%%%%%%%%%%%%%%%%%%%%%%%%%%%%%%%%%%%%%%%%%%%%%%


\begin{frame}{}

{\LARGE
\begin{center}
Find the $x$-coordinates of all local maxima of the function
$$
F(x)=\int_1^x e^t(t+1)(t-1)(t-3)^2dt.
$$
\end{center}}

\end{frame}


%%%%%%%%%%%%%%%%%%%%%%%%%%%%%%%%%%%%%%%%%%%%%%%%%%%%%%


\begin{frame}{}


{\Large
\begin{center}
Answer 3

\pause

\vspace{1.5cm}

Answer 2

\pause 

\vspace{1.5cm}

Answer 1
\end{center}}

  
\end{frame}


%%%%%%%%%%%%%%%%%%%%%%%%%%%%%%%%%%%%%%%%%%%%%%%%%%%%%%


\begin{frame}{}


{\LARGE
\begin{center}
    $x=-1$
\end{center}}

\end{frame}





\end{document}
