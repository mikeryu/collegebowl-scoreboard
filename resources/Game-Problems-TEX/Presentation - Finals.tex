 \documentclass[11pt]{beamer}
\mode<presentation>
%\includeonlyframes{yep}


%% Packages
\usepackage{amsmath,amssymb,amsthm,amsfonts,graphicx,url,colordvi,bbm}
\usepackage{graphics,graphics,latexsym,multicol,epsfig}
\usepackage{enumerate,url}
\usepackage{wasysym}
 \usepackage{vwcol} 
\usepackage{pifont}
\usepackage{cancel}
\usepackage{empheq}

%% Color Definitions
% These are set with RGB codes, which you can for sure find online, but Macs have a cool app called Digital 
% Color Meter which is super helpful. I tend to give generic names so that I can just change the numbers
% when I want to change colors.
\definecolor{dark}{RGB}{30, 0, 70}
\definecolor{medium}{RGB}{129, 0, 49}
\definecolor{light}{RGB}{0, 0, 200}
\definecolor{accent}{RGB}{178,153,108}
\definecolor{ivory}{RGB}{255,255,255}


% \definecolor{ivory}{RGB}{250,240,225}


% \definecolor{accent}{RGB}{208,120,149}
% 178, 153, 108

%% Theme
% This is the style and layout. I use an outer theme that gets rid of slide numbers, which is just a style file 
% that you need to include wherever you've compiled your slides. I've attached it to the email.
% The navigation symbols are those things in the bottom right corner, but I don't like them so I set them to blank.
% \usetheme{Madrid}
% \usetheme{AnnArbor}
% \usetheme{Boadilla}
% \usetheme{Copenhagen}
% \usetheme{Darmstadt}
% \usetheme{Berkeley}
% \usetheme{Dresden}
% \usetheme{Frankfurt}
\usetheme{JuanLesPins}


\setbeamertemplate{navigation symbols}{}
\useoutertheme{noslidenum}
\setbeamersize{text margin left=15pt,text margin right=15pt}


%% Color Sets
% this is where I custom set the colors of all my objects. Beamer is weird in that most objects have a foreground (fg) color 
% and a background (bg) color. You can also do color mixing with exclamation points, so like the background canvas is 25% 
% white and 75% ivory.

\usecolortheme[named=medium]{structure}
\setbeamertemplate{enumerate items}[square]
\setbeamercolor{item projected}{bg=accent,fg=white}
\setbeamertemplate{itemize items}{\textcolor{accent}{$\RHD$}}
\setbeamertemplate{qed symbol}{\textcolor{accent}{$\darksquare$}}
\setbeamercolor{background canvas}{bg=ivory!10!white}
\setbeamercolor{author}{fg=dark}
\setbeamercolor{date}{fg=dark}
\setbeamercolor{institute}{fg=accent}
\setbeamercolor{normal text}{fg=dark}
\setbeamercolor{alerted text}{fg=accent}
\setbeamercolor{title}{bg=medium}
\setbeamercolor{title}{fg=white}
\setbeamercolor{block title}{bg=accent}
\setbeamercolor{block title}{fg=white}
\setbeamercolor{block body}{bg=white}
\setbeamercolor{section in head/foot}{fg=accent}
\setbeamercolor{institute in head/foot}{fg=accent}
\setbeamercolor{author in head/foot}{fg=accent}
\setbeamercolor{date in head/foot}{fg=accent}


%% Tikz
\usepackage{pgf,tikz}
\usetikzlibrary{arrows,snakes}
\usetikzlibrary{calc}

%% Math Commands
\newtheorem{conjecture}[theorem]{Conjecture}
\newtheorem{proposition}[theorem]{Proposition}
\newtheorem{remark}[theorem]{Remark}
\newtheorem{claim}{Claim}
\renewcommand{\L}{\mathcal{L}}
\newcommand*\widefbox[1]{\fbox{\hspace{1em}#1\hspace{1em}}}
\newcommand{\la}{\langle}
\newcommand{\ra}{\rangle}
\newcommand{\N}{\mathbb{N}}
\newcommand{\norm}[1]{\left\lVert#1\right\rVert}
\newcommand{\abs}[1]{|#1|}
\newcommand{\Abs}[1]{\big{|}#1\big{|}}
\newcommand{\ABS}[1]{\left|#1\right|}
\newcommand{\mch}{\mathcal{H}}
\def\({\left(}
\def\){\right)}

%% Font
% Fonts are complicated in Beamer, I always just end up googling it and copying/pasting
% the code they have listed. You can certainly delete this chunk if you want to use the default.
% \usepackage[english]{babel}
% %\usefonttheme{serif}
% \usepackage[light,math]{kurier}
% \usepackage[T1]{fontenc}
% \fontsize{11}{6}

\title[]{{11-12 College Bowl Final Round}}
\author[]{{36th Annual Mathematics Field Day}}
\institute{Westmont College}
\date{2025}

\begin{document}

%%%%%%%%%%%%%%%%%%%%%%%%%%%%%%%%%%%%%%%%%%%%%%%%%%%%%%

\begin{frame}
  \titlepage
\end{frame}











%%%%%%%%%%%%%%%%%%%%%%%%%%%%%%%%%%%%%%%%%%%%%%%%%%%%%% Question 1

\begin{frame}{}


{\LARGE
\begin{center}
Toss-Up Question
\end{center}}


 
\end{frame}

%%%%%%%%%%%%%%%%%%%%%%%%%%%%%%%%%%%%%%%%%%%%%%%%%%%%%%



\begin{frame}{}

{\LARGE
\begin{center}
    The number $2025$ is a perfect square. If an integer is chosen between $1$ and $2025$, inclusive, what is the probability it will also be a perfect square?
\end{center}}

\end{frame}

%%%%%%%%%%%%%%%%%%%%%%%%%%%%%%%%%%%%%%%%%%%%%%%%%%%%%%



\begin{frame}{}

{\Large
\begin{center}
Answer 3

\pause

\vspace{1.5cm}

Answer 2

\pause 

\vspace{1.5cm}

Answer 1
\end{center}}

  
\end{frame}

%%%%%%%%%%%%%%%%%%%%%%%%%%%%%%%%%%%%%%%%%%%%%%%%%%%%%%


\begin{frame}{}


{\LARGE
\begin{center}
    $\frac{1}{45}$
\end{center}}

\end{frame}



%%%%%%%%%%%%%%%%%%%%%%%%%%%%%%%%%%%%%%%%%%%%%%%%%%%%%%


\begin{frame}{}

{\LARGE
\begin{center}
Follow-Up Question
\end{center}}

\end{frame}


%%%%%%%%%%%%%%%%%%%%%%%%%%%%%%%%%%%%%%%%%%%%%%%%%%%%%%


\begin{frame}{}

{\LARGE
\begin{center}
    Two six-sided dice with sides labeled one through six are rolled. What is the probability that the sum of the numbers rolled is either a perfect square or a multiple of $4$?
\end{center}}

\end{frame}


%%%%%%%%%%%%%%%%%%%%%%%%%%%%%%%%%%%%%%%%%%%%%%%%%%%%%%


\begin{frame}{}

{\Large
\begin{center}
Answer 3

\pause

\vspace{1.5cm}

Answer 2

\pause 

\vspace{1.5cm}

Answer 1
\end{center}}

  
\end{frame}


%%%%%%%%%%%%%%%%%%%%%%%%%%%%%%%%%%%%%%%%%%%%%%%%%%%%%%


\begin{frame}{}


{\LARGE
\begin{center}
    $\frac{13}{36}$
\end{center}}

\end{frame}


%%%%%%%%%%%%%%%%%%%%%%%%%%%%%%%%%%%%%%%



%%%%%%%%%%%%%%%%%%%%%%%%%%%%%%%%%%%%%%%%%%%%%%%%%%%%%% Question 2

\begin{frame}{}


{\LARGE
\begin{center}
Toss-Up Question
\end{center}}




 
\end{frame}

%%%%%%%%%%%%%%%%%%%%%%%%%%%%%%%%%%%%%%%%%%%%%%%%%%%%%%



\begin{frame}{}

{\LARGE
\begin{center}
   Find the largest value of $x$ at which the function $f(x)=\frac{x^2-x-2}{x^2+x-6}$ has a vertical asymptote.
\end{center}}

\end{frame}

%%%%%%%%%%%%%%%%%%%%%%%%%%%%%%%%%%%%%%%%%%%%%%%%%%%%%%



\begin{frame}{}

{\Large
\begin{center}
Answer 3

\pause

\vspace{1.5cm}

Answer 2

\pause 

\vspace{1.5cm}

Answer 1
\end{center}}

  
\end{frame}

%%%%%%%%%%%%%%%%%%%%%%%%%%%%%%%%%%%%%%%%%%%%%%%%%%%%%%


\begin{frame}{}


{\LARGE
\begin{center}
     $x=-3$
\end{center}}

\end{frame}



%%%%%%%%%%%%%%%%%%%%%%%%%%%%%%%%%%%%%%%%%%%%%%%%%%%%%%


\begin{frame}{}

{\LARGE
\begin{center}
Follow-Up Question
\end{center}}

\end{frame}


%%%%%%%%%%%%%%%%%%%%%%%%%%%%%%%%%%%%%%%%%%%%%%%%%%%%%%


\begin{frame}{}

{\LARGE
\begin{center}
    Find the coefficient of the cubic term of $(2x-1)^8$.
\end{center}}

\end{frame}


%%%%%%%%%%%%%%%%%%%%%%%%%%%%%%%%%%%%%%%%%%%%%%%%%%%%%%


\begin{frame}{}

{\Large
\begin{center}
Answer 3

\pause

\vspace{1.5cm}

Answer 2

\pause 

\vspace{1.5cm}

Answer 1
\end{center}}

  
\end{frame}


%%%%%%%%%%%%%%%%%%%%%%%%%%%%%%%%%%%%%%%%%%%%%%%%%%%%%%


\begin{frame}{}


{\LARGE
\begin{center}
    $-448$
\end{center}}

\end{frame}


%%%%%%%%%%%%%%%%%%%%%%%%%%%%%%%%%%%%%%%%%%%%%%%%%%%%%% Question 3
\begin{frame}{}


{\LARGE
\begin{center}
Toss-Up Question
\end{center}}





 
\end{frame}

%%%%%%%%%%%%%%%%%%%%%%%%%%%%%%%%%%%%%%%%%%%%%%%%%%%%%%



\begin{frame}{}

{\LARGE
\begin{center}
Simplify $\log_23\cdot\log_34\cdot\log_48$.
\end{center}}

\end{frame}

%%%%%%%%%%%%%%%%%%%%%%%%%%%%%%%%%%%%%%%%%%%%%%%%%%%%%%



\begin{frame}{}

{\Large
\begin{center}
Answer 3

\pause

\vspace{1.5cm}

Answer 2

\pause 

\vspace{1.5cm}

Answer 1
\end{center}}

  
\end{frame}

%%%%%%%%%%%%%%%%%%%%%%%%%%%%%%%%%%%%%%%%%%%%%%%%%%%%%%


\begin{frame}{}


{\LARGE
\begin{center}
    $3$
\end{center}}

\end{frame}



%%%%%%%%%%%%%%%%%%%%%%%%%%%%%%%%%%%%%%%%%%%%%%%%%%%%%%


\begin{frame}{}

{\LARGE
\begin{center}
Follow-Up Question
\end{center}}

\end{frame}


%%%%%%%%%%%%%%%%%%%%%%%%%%%%%%%%%%%%%%%%%%%%%%%%%%%%%%


\begin{frame}{}

{\LARGE
\begin{center}
    Find the zeros of $f(x)=\log_{25}x+\log_5(\sqrt{x}+1)$.
\end{center}}

\end{frame}


%%%%%%%%%%%%%%%%%%%%%%%%%%%%%%%%%%%%%%%%%%%%%%%%%%%%%%


\begin{frame}{}

{\Large
\begin{center}
Answer 3

\pause

\vspace{1.5cm}

Answer 2

\pause 

\vspace{1.5cm}

Answer 1
\end{center}}

  
\end{frame}


%%%%%%%%%%%%%%%%%%%%%%%%%%%%%%%%%%%%%%%%%%%%%%%%%%%%%%


\begin{frame}{}

{\LARGE
\begin{center}
    $\frac{3\pm\sqrt{5}}{2}$
\end{center}}

\end{frame}

%%%%%%%%%%%%%%%%%%%%%%%%%%%%%%%%%%%%%%%%%%%%%%%%%%%%%%


%%%%%%%%%%%%%%%%%%%%%%%%%%%%%%%%%%%%%%%%%%%%%%%%%%%%%% Question 4
\begin{frame}{}


{\LARGE
\begin{center}
Toss-Up Question
\end{center}}


 
\end{frame}

%%%%%%%%%%%%%%%%%%%%%%%%%%%%%%%%%%%%%%%%%%%%%%%%%%%%%%



\begin{frame}{}

{\LARGE
\begin{center}
Find the equations of the two vertical tangent lines to $(x-2)^2+(y+3)^2=25$.
\end{center}}

\end{frame}

%%%%%%%%%%%%%%%%%%%%%%%%%%%%%%%%%%%%%%%%%%%%%%%%%%%%%%



\begin{frame}{}

{\Large
\begin{center}
Answer 3

\pause

\vspace{1.5cm}

Answer 2

\pause 

\vspace{1.5cm}

Answer 1
\end{center}}

  
\end{frame}

%%%%%%%%%%%%%%%%%%%%%%%%%%%%%%%%%%%%%%%%%%%%%%%%%%%%%%


\begin{frame}{}


{\LARGE
\begin{center}
    $x=-3, x=7$
\end{center}}

\end{frame}



%%%%%%%%%%%%%%%%%%%%%%%%%%%%%%%%%%%%%%%%%%%%%%%%%%%%%%


\begin{frame}{}

{\LARGE
\begin{center}
Follow-Up Question
\end{center}}

\end{frame}


%%%%%%%%%%%%%%%%%%%%%%%%%%%%%%%%%%%%%%%%%%%%%%%%%%%%%%


\begin{frame}{}

{\LARGE
\begin{center}
    How many times do the graphs of $y=\sin x$ and $y=\frac{x}{8\pi}$ intersect?
\end{center}}

\end{frame}


%%%%%%%%%%%%%%%%%%%%%%%%%%%%%%%%%%%%%%%%%%%%%%%%%%%%%%


\begin{frame}{}

{\Large
\begin{center}
Answer 3

\pause

\vspace{1.5cm}

Answer 2

\pause 

\vspace{1.5cm}

Answer 1
\end{center}}

  
\end{frame}


%%%%%%%%%%%%%%%%%%%%%%%%%%%%%%%%%%%%%%%%%%%%%%%%%%%%%%


\begin{frame}{}


{\LARGE
\begin{center}
    $15$ times
\end{center}}

\end{frame}



%%%%%%%%%%%%%%%%%%%%%%%%%%%%%%%%%%%%%%%%%%%%%%%%%%%%%% Question 5
\begin{frame}{}


{\LARGE
\begin{center}
Toss-Up Question
\end{center}}


 
\end{frame}

%%%%%%%%%%%%%%%%%%%%%%%%%%%%%%%%%%%%%%%%%%%%%%%%%%%%%%



\begin{frame}{}

{\LARGE
\begin{center}
How many distinct digits can be the last (units) digit of a perfect square?
\end{center}}

\end{frame}

%%%%%%%%%%%%%%%%%%%%%%%%%%%%%%%%%%%%%%%%%%%%%%%%%%%%%%



\begin{frame}{}

{\Large
\begin{center}
Answer 3

\pause

\vspace{1.5cm}

Answer 2

\pause 

\vspace{1.5cm}

Answer 1
\end{center}}

  
\end{frame}

%%%%%%%%%%%%%%%%%%%%%%%%%%%%%%%%%%%%%%%%%%%%%%%%%%%%%%


\begin{frame}{}


{\LARGE
\begin{center}
    $6$
\end{center}}

\end{frame}



%%%%%%%%%%%%%%%%%%%%%%%%%%%%%%%%%%%%%%%%%%%%%%%%%%%%%%


\begin{frame}{}

{\LARGE
\begin{center}
Follow-Up Question
\end{center}}

\end{frame}


%%%%%%%%%%%%%%%%%%%%%%%%%%%%%%%%%%%%%%%%%%%%%%%%%%%%%%


\begin{frame}{}

{\LARGE
\begin{center}
    The average age of a group of students and teachers is $22$. The students' average age is $15$. The teachers' average age is $36$. Find the ratio of students to teachers. 
\end{center}}

\end{frame}


%%%%%%%%%%%%%%%%%%%%%%%%%%%%%%%%%%%%%%%%%%%%%%%%%%%%%%


\begin{frame}{}

{\Large
\begin{center}
Answer 3

\pause

\vspace{1.5cm}

Answer 2

\pause 

\vspace{1.5cm}

Answer 1
\end{center}}

  
\end{frame}


%%%%%%%%%%%%%%%%%%%%%%%%%%%%%%%%%%%%%%%%%%%%%%%%%%%%%%


\begin{frame}{}


{\LARGE
\begin{center}
    $2:1$ OR $2$ to $1$
\end{center}}

\end{frame}



%%%%%%%%%%%%%%%%%%%%%%%%%%%%%%%%%%%%%%%%%%%%%%%%%%%%%% Question 6
\begin{frame}{}


{\LARGE
\begin{center}
Toss-Up Question
\end{center}}



 
\end{frame}

%%%%%%%%%%%%%%%%%%%%%%%%%%%%%%%%%%%%%%%%%%%%%%%%%%%%%%



\begin{frame}{}



{\LARGE
\begin{center}
Let $p(x)=ax^3+bx^2+cx+d$. If $3i$ and $1$ are two roots of $p(x)$, find $a$, $b$, $c$, and $d$.
\end{center}}


\end{frame}

%%%%%%%%%%%%%%%%%%%%%%%%%%%%%%%%%%%%%%%%%%%%%%%%%%%%%%



\begin{frame}{}

{\Large
\begin{center}
Answer 3

\pause

\vspace{1.5cm}

Answer 2

\pause 

\vspace{1.5cm}

Answer 1
\end{center}}

  
\end{frame}

%%%%%%%%%%%%%%%%%%%%%%%%%%%%%%%%%%%%%%%%%%%%%%%%%%%%%%


\begin{frame}{}

{\LARGE
\begin{center}
    $a=1, b=-1, c=9, d=-9$
\end{center}}

\end{frame}



%%%%%%%%%%%%%%%%%%%%%%%%%%%%%%%%%%%%%%%%%%%%%%%%%%%%%%


\begin{frame}{}

{\LARGE
\begin{center}
Follow-Up Question
\end{center}}

\end{frame}


%%%%%%%%%%%%%%%%%%%%%%%%%%%%%%%%%%%%%%%%%%%%%%%%%%%%%%


\begin{frame}{}

{\LARGE
\begin{center}
For what values of $b$ will the function $f(x)=2x^3+bx^2-4x^2+3x-2bx-6$ have only one real root? 
\end{center}}

\end{frame}


%%%%%%%%%%%%%%%%%%%%%%%%%%%%%%%%%%%%%%%%%%%%%%%%%%%%%%


\begin{frame}{}

{\Large
\begin{center}
Answer 3

\pause

\vspace{1.5cm}

Answer 2

\pause 

\vspace{1.5cm}

Answer 1
\end{center}}

  
\end{frame}


%%%%%%%%%%%%%%%%%%%%%%%%%%%%%%%%%%%%%%%%%%%%%%%%%%%%%%


\begin{frame}{}


{\LARGE
\begin{center}
    $-2\sqrt{6}<b<2\sqrt{6}$
\end{center}}

\end{frame}

%%%%%%%%%%%%%%%%%%%%%%%%%%%%%%%%%%%%%%%%%%%%%%%%%%%%%%


%%%%%%%%%%%%%%%%%%%%%%%%%%%%%%%%%%%%%%%%%%%%%%%%%%%%%% Question 7
\begin{frame}{}


{\LARGE
\begin{center}
Toss-Up Question
\end{center}}


 
\end{frame}

%%%%%%%%%%%%%%%%%%%%%%%%%%%%%%%%%%%%%%%%%%%%%%%%%%%%%%



\begin{frame}{}

{\LARGE
\begin{center}
A certain town has coins with 1 cent, 2 cent, and 5 cent values. What is the smallest number of cents which cannot be made using at most three of these coins?
\end{center}}

\end{frame}

%%%%%%%%%%%%%%%%%%%%%%%%%%%%%%%%%%%%%%%%%%%%%%%%%%%%%%



\begin{frame}{}

{\Large
\begin{center}
Answer 3

\pause

\vspace{1.5cm}

Answer 2

\pause 

\vspace{1.5cm}

Answer 1
\end{center}}

  
\end{frame}

%%%%%%%%%%%%%%%%%%%%%%%%%%%%%%%%%%%%%%%%%%%%%%%%%%%%%%


\begin{frame}{}


{\LARGE
\begin{center}
    $13$
\end{center}}

\end{frame}



%%%%%%%%%%%%%%%%%%%%%%%%%%%%%%%%%%%%%%%%%%%%%%%%%%%%%%


\begin{frame}{}

{\LARGE
\begin{center}
Follow-Up Question
\end{center}}

\end{frame}


%%%%%%%%%%%%%%%%%%%%%%%%%%%%%%%%%%%%%%%%%%%%%%%%%%%%%%


\begin{frame}{}

{\LARGE
\begin{center}
Circle $\mathcal{O}$ has radius $5\sqrt{3}$. Sector $A\mathcal{O}B$ has a central angle of measure $\frac{2\pi}{3}$. Find the length of segment $\overline{AB}$. 
\end{center}}

\end{frame}


%%%%%%%%%%%%%%%%%%%%%%%%%%%%%%%%%%%%%%%%%%%%%%%%%%%%%%


\begin{frame}{}

{\Large
\begin{center}
Answer 3

\pause

\vspace{1.5cm}

Answer 2

\pause 

\vspace{1.5cm}

Answer 1
\end{center}}

  
\end{frame}


%%%%%%%%%%%%%%%%%%%%%%%%%%%%%%%%%%%%%%%%%%%%%%%%%%%%%%


\begin{frame}{}


{\LARGE
\begin{center}
    $15$ units
\end{center}}

\end{frame}


%%%%%%%%%%%%%%%%%%%%%%%%%%%%%%%%%%%%%%%%%%%%%%%%%%%%%% Question 8
\begin{frame}{}


{\LARGE
\begin{center}
Toss-Up Question
\end{center}}


 
\end{frame}

%%%%%%%%%%%%%%%%%%%%%%%%%%%%%%%%%%%%%%%%%%%%%%%%%%%%%%



\begin{frame}{}

{\LARGE
\begin{center}
Simplify $2\cos^2(\frac{\pi}{12})-1$.
\end{center}}

\end{frame}

%%%%%%%%%%%%%%%%%%%%%%%%%%%%%%%%%%%%%%%%%%%%%%%%%%%%%%



\begin{frame}{}

{\Large
\begin{center}
Answer 3

\pause

\vspace{1.5cm}

Answer 2

\pause 

\vspace{1.5cm}

Answer 1
\end{center}}

  
\end{frame}

%%%%%%%%%%%%%%%%%%%%%%%%%%%%%%%%%%%%%%%%%%%%%%%%%%%%%%


\begin{frame}{}


{\LARGE
\begin{center}
    $\frac{\sqrt{3}}{2}$
\end{center}}

\end{frame}



%%%%%%%%%%%%%%%%%%%%%%%%%%%%%%%%%%%%%%%%%%%%%%%%%%%%%%


\begin{frame}{}

{\LARGE
\begin{center}
Follow-Up Question
\end{center}}

\end{frame}


%%%%%%%%%%%%%%%%%%%%%%%%%%%%%%%%%%%%%%%%%%%%%%%%%%%%%%


\begin{frame}{}

{\LARGE
\begin{center}
Simplify 
    $$
    \csc^2\left(\frac{\pi}{8}\right)\csc^2\left(\frac{3\pi}{8}\right).
    $$
\end{center}}

\end{frame}


%%%%%%%%%%%%%%%%%%%%%%%%%%%%%%%%%%%%%%%%%%%%%%%%%%%%%%


\begin{frame}{}

{\Large
\begin{center}
Answer 3

\pause

\vspace{1.5cm}

Answer 2

\pause 

\vspace{1.5cm}

Answer 1
\end{center}}

  
\end{frame}


%%%%%%%%%%%%%%%%%%%%%%%%%%%%%%%%%%%%%%%%%%%%%%%%%%%%%%


\begin{frame}{}


{\LARGE
\begin{center}
    $8$
\end{center}}

\end{frame}



%%%%%%%%%%%%%%%%%%%%%%%%%%%%%%%%%%%%%%%%%%%%%%%%%%%%%% Question 9
\begin{frame}{}


{\LARGE
\begin{center}
Toss-Up Question
\end{center}}


 
\end{frame}

%%%%%%%%%%%%%%%%%%%%%%%%%%%%%%%%%%%%%%%%%%%%%%%%%%%%%%



\begin{frame}{}

{\LARGE
\begin{center}
How many $2\times 2 \times 2$ cubes can be fit in an $8\times 8\times 8$ cube?
\end{center}}

\end{frame}

%%%%%%%%%%%%%%%%%%%%%%%%%%%%%%%%%%%%%%%%%%%%%%%%%%%%%%



\begin{frame}{}

{\Large
\begin{center}
Answer 3

\pause

\vspace{1.5cm}

Answer 2

\pause 

\vspace{1.5cm}

Answer 1
\end{center}}

  
\end{frame}

%%%%%%%%%%%%%%%%%%%%%%%%%%%%%%%%%%%%%%%%%%%%%%%%%%%%%%


\begin{frame}{}


{\LARGE
\begin{center}
    $64$
\end{center}}

\end{frame}



%%%%%%%%%%%%%%%%%%%%%%%%%%%%%%%%%%%%%%%%%%%%%%%%%%%%%%


\begin{frame}{}

{\LARGE
\begin{center}
Follow-Up Question
\end{center}}

\end{frame}


%%%%%%%%%%%%%%%%%%%%%%%%%%%%%%%%%%%%%%%%%%%%%%%%%%%%%%


\begin{frame}{}

{\LARGE
\begin{center}
    How many counting numbers under $10,000$ have an odd number of factors?
\end{center}}

\end{frame}


%%%%%%%%%%%%%%%%%%%%%%%%%%%%%%%%%%%%%%%%%%%%%%%%%%%%%%


\begin{frame}{}

{\Large
\begin{center}
Answer 3

\pause

\vspace{1.5cm}

Answer 2

\pause 

\vspace{1.5cm}

Answer 1
\end{center}}

  
\end{frame}


%%%%%%%%%%%%%%%%%%%%%%%%%%%%%%%%%%%%%%%%%%%%%%%%%%%%%%


\begin{frame}{}


{\LARGE
\begin{center}
    $99$
\end{center}}

\end{frame}



%%%%%%%%%%%%%%%%%%%%%%%%%%%%%%%%%%%%%%%%%%%%%%%%%%%%%% Question 10
\begin{frame}{}


{\LARGE
\begin{center}
Toss-Up Question
\end{center}}


 
\end{frame}

%%%%%%%%%%%%%%%%%%%%%%%%%%%%%%%%%%%%%%%%%%%%%%%%%%%%%%



\begin{frame}{}



{\LARGE
\begin{center}
 Find the vertex of the graph determined by the equation $y=3-2|x-2|$.
\end{center}}


\end{frame}

%%%%%%%%%%%%%%%%%%%%%%%%%%%%%%%%%%%%%%%%%%%%%%%%%%%%%%



\begin{frame}{}

{\Large
\begin{center}
Answer 3

\pause

\vspace{1.5cm}

Answer 2

\pause 

\vspace{1.5cm}

Answer 1
\end{center}}

  
\end{frame}

%%%%%%%%%%%%%%%%%%%%%%%%%%%%%%%%%%%%%%%%%%%%%%%%%%%%%%


\begin{frame}{}

{\LARGE
\begin{center}
    $(2,3)$
\end{center}}

\end{frame}



%%%%%%%%%%%%%%%%%%%%%%%%%%%%%%%%%%%%%%%%%%%%%%%%%%%%%%


\begin{frame}{}

{\LARGE
\begin{center}
Follow-Up Question
\end{center}}

\end{frame}


%%%%%%%%%%%%%%%%%%%%%%%%%%%%%%%%%%%%%%%%%%%%%%%%%%%%%%


\begin{frame}{}

{\LARGE
\begin{center}
Find the values of $x$ for which the graph determined by the equation $y=3-2|x-2|$ is below the $x$-axis. 
\end{center}}

\end{frame}


%%%%%%%%%%%%%%%%%%%%%%%%%%%%%%%%%%%%%%%%%%%%%%%%%%%%%%


\begin{frame}{}

{\Large
\begin{center}
Answer 3

\pause

\vspace{1.5cm}

Answer 2

\pause 

\vspace{1.5cm}

Answer 1
\end{center}}

  
\end{frame}


%%%%%%%%%%%%%%%%%%%%%%%%%%%%%%%%%%%%%%%%%%%%%%%%%%%%%%


\begin{frame}{}


{\LARGE
\begin{center}
    $x<\frac{1}{2}$ or $x>\frac{7}{2}$ OR $x<.5$ or $x>3.5$
\end{center}}

\end{frame}

%%%%%%%%%%%%%%%%%%%%%%%%%%%%%%%%%%%%%%%%%%%%%%%%%%%%%%




\end{document}
